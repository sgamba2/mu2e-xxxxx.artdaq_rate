% -*- mode:flyspell; mode:latex -*-
\documentclass[12pt]{article}

% \addtolength{\oddsidemargin} {-0.885in}
% \addtolength{\textwidth}{1.75in}
% \addtolength{\evensidemargin}{-0.8in}
% topmargin -0.5in
\usepackage[a4paper, top=1cm, left=1.5cm, right=1.5cm]{geometry} % width= , 

\usepackage[latin1]{inputenc}
\usepackage[T1]{fontenc}
\usepackage[english]{babel}
\usepackage{graphicx}
\usepackage{float}


\usepackage{tikz}
\usepackage{[caption}
\usetikzlibrary{arrows}
\usetikzlibrary{decorations.markings}
%\usepackage{lineno}

\usetikzlibrary{decorations.pathmorphing}
% \usepackage[absolute,overlay]{textpos}
% \usepackage{onimage}

\usepackage{times}
\usepackage{graphics}

% \usepackage{subfigure}
% \usepackage{scalefnt}
%
% \renewcommand\thesubfigure{\arabic{subfigure}}

\usepackage{amsmath}
\usepackage{hyperref}
\usepackage{hhline}
\usepackage{subfig}
\usepackage{color}
\usepackage[all]{hypcap}

\usepackage[normalem]{ulem}  % for striking out
% \usepackage{fancyhdr}
% \pagestyle{fancy}
% \fancyhead[C]{}
% \fancyhead[L] {\it{Mu2e-doc-29670-v1.0} }
%%%%%%%%%%%%%%%%%%%%%%%%%%%%%%%%%%%%%%%%%%%%%%%%%%%%%%%%%%%%%%%%%%%%%%%%%%%%%%
% use natbib - biblatex not available on Mu2e interactive nodes
%%%%%%%%%%%%%%%%%%%%%%%%%%%%%%%%%%%%%%%%%%%%%%%%%%%%%%%%%%%%%%%%%%%%%%%%%%%%%%
\usepackage[square,sort,comma,numbers]{natbib}

% location of the .bib files: env var BIBINPUTS (~/library/bibliography)

% \usepackage[backend=biber, style=numeric-comp, sorting=ynt] {biblatex}
% \addbibresource{clfv.bib}

% \addbibresource{stntuple.bib}
% \addbibresource{mu2e_web.bib}
% \addbibresource{radiative_pion_capture.bib}

\graphicspath{{figures/}}
%%%%%%%%%%%%%%%%%%%%%%%%%%%%%%%%%%%%%%%%%%%%%%%%%%%%%%%%%%%%%%%%%%%%%%%%%%%%%%
% for portability, make sure all commands are included locally
% order them alphabetically
%%%%%%%%%%%%%%%%%%%%%%%%%%%%%%%%%%%%%%%%%%%%%%%%%%%%%%%%%%%%%%%%%%%%%%%%%%%%%%
% \include{commands}

\newcommand {\keVc}       {\mbox{$\rm keV\!/\!c$}}
\newcommand {\kmax}       {\mbox{$k_{\rm max}$}}

\newcommand {\MeVc}       {\mbox{$\rm MeV\!/c$}}
\newcommand {\MeVcsq}     {\mbox{$\rm MeV\!/c^2$}}

\newcommand {\mumemconv}[1][A] {\mbox{$\mu^- \textrm{#1} \rightarrow e^- \textrm{#1}$}}
% Define a relay to have 2 default arguments instead of limit of 1
\newcommand {\mumepconv}[1][A] {%
  \def\ArgI{{#1}}%store the first argument
  \mumepconvRelay
}
\newcommand \mumepconvRelay[1][A]  {\mbox{$\mu^- \textrm{\ArgI} \rightarrow e^+ \textrm{#1}$}}
\newcommand {\muminus}    {\mbox{$\mu^-$}}
\newcommand {\muplus}    {\mbox{$\mu^+$}}
\newcommand {\MuToEm}     {\mbox{$\mu^- \ra e^-$}}
\newcommand {\MuToEp}     {\mbox{$\mu^- \ra e^+$}}
\newcommand {\MuPToEp}    {\mbox{$\mu^+ \ra e^+$}}
\newcommand {\ra}        {\rightarrow}
\newcommand {\tandip}    {\mbox{$\tan \lambda$}}

\newcommand {\Pb}[1]     {\mbox{$\rm ^{#1}Pb$}}                 % isotopes of lead
\newcommand {\Au}[1]     {\mbox{$\rm ^{#1}Au$}}                 % isotopes of gold
\newcommand {\Ir}[1]     {\mbox{$\rm ^{#1}Ir$}}                 % isotopes of iridium
%%%%%%%%%%%%%%%%%%%%%%%%%%%%%%%%%%%%%%%%%%%%%%%%%%%%%%%%%%%%%%%%%%%%%%%%%%%%%%
% editing commands
%%%%%%%%%%%%%%%%%%%%%%%%%%%%%%%%%%%%%%%%%%%%%%%%%%%%%%%%%%%%%%%%%%%%%%%%%%%%%%
\newcommand {\add}[1]    {{\red #1}}
\newcommand {\alt}[1]    {{\green #1}} %alternate comment color
\newcommand {\del}[1]    {{\blue \sout{#1}}}
\newcommand {\dlt}[1]    {{\violet \sout{#1}}} %alternate delete color

\newcommand {\black}     {\color{black}}
\newcommand {\red}       {\color{red}}
\newcommand {\blue}      {\color{blue}}
\newcommand {\strike}[1] {{\blue \sout{#1}}}
%%%%%%%%%%%%%%%%%%%%%%%%%%%%%%%%%%%%%%%%%%%%%%%%%%%%%%%%%%%%%%%%%%%%%%%%%%%%%%
\begin{document}

\begin{titlepage}
  \begin{flushright}
    \bf {MU2E/PHYSICS/xxxxx} \\
    version 1.0
    \today
 \end{flushright}

  \vspace{1cm}

  \begin{center}
    {\Large \bf Commissioning of the Mu2e Data AcQuisition system and the Vertical Slice Test of the straw tracker

      \vspace{0.3in}

      11. An artdaq Demo tests
    }

    \vspace{1cm}
    Pavel Murat (FNAL) \\
    Sara Gamba (University of Pisa)
%     S. Gamba  \footnote{\texttt{Fermilab; e-mail:s.gamba2\@studenti.unipi.it} (University of Pisa)
%     P. Murat \footnote{\texttt{Fermilab; e-mail: murat\@fnal.gov}

   
    version 1.0
    \today
 \end{center}

  \begin{abstract}
    This note presents the initial results of an analysis on the artdaq rate.
    \vspace{0.2in}
  \end{abstract}

\end{titlepage}
% \frontmatter
% \chapter*{Abstract}
%
% \addcontentsline{toc}{chapter}{Abstract}
%
% \mainmatter
%
{\tableofcontents}
%%%%%%%%%%%%%%%%%%%%%%%%%%%%%%%%%%%%%%%%%%%%%%%%%%%%%%%%%%%%%%%%%%%%%%%%%%%%%%%
%\chapter{Calibration}
%%%%%%%%%%%%%%%%%%%%%%%%%%%%%%%%%%%%%%%%%%%%%%%%%%%%%%%%%%%%%%%%%%%%%%%%%%%%%%%
% \input{input_data}

%%%%%%%%%%%%%%%%%%%%%%%%%%%%%%%%%%%%%%%%%%%%%%%%%%%%%%%%%%%%%%%%%%%%%%%%%%%%%%%
\newpage

\section {Notes for the authors}
\subsection {Revision history} 
\begin{itemize}
\item
  v1.01: initial version
\end{itemize}
\newpage
\section{Running an artdaq demo}
We were running an artdaq demo simulation to test if the artdaq supports high event rates: we should be able to get data at a rate of more than 200MB/s. This demo is called ToySim and it is taken from $ots\_mu2e\_tracker$, adding a $Generators$ directory. This directory is very similar to $artdaq\_demo\/artdaq-demo\/Generators$ directory. We were sending some random events to two boardreaders. These events were taken by two eventbuilders that have a common data logger. A dispatcher, which aggregates DQM metrics and presents them to a visualizer application, was used. We were trying to run the demo changing some parameters:
\begin{itemize}
\item time between events: the function called $FillBuffer(buffer,bytes\_read)$ is called every time after $throttle\_ usecs \ \mu$s;
\item size of the events: $nADCchannels$ is the variable we are changing and it is used to define how many bytes are readout ($bytes\_read = sizeof(demo::ToyFragment::Header)$ $+ nADCchannels \times sizeof(data\_t)$, so the dimension will be this number $\times$ 2Bytes more or less);
\item number of eventbuilders (1 or 2);
\item number of boardreaders (1 or 2);
\item the presence of a dispatcher;
\item the  transferPluginType: $Shmem$ and  $TCPSocket$.
\end{itemize} 
If the artdaq framework is able to process this events, we should see a $GetNext$ Frequency equal to the inverse of the $throttle\_ usecs \ \mu$s value. We have tried to change the variables $nADCchannels$ and $throttle\_usecs$.
\section{Results}
During the running some errors appeared, as the following:
\begin{itemize}
    \item Bad Omen: Data Buffer has exceeded its size limits.
    \\
    ($seq\_id$=125, $frag\_id$=0, frags=1001/1000, szB=200248048/1048576000), timestamps=124-1124
    \\
    with this type of error the rate is not stable;
    \item Back-pressure condition: All Shared Memory buffers have been full for 12.025 s!
\end{itemize}
 We report some tables that show our results. 
\begin{center}  
\begin{table}[!h]
\centering
\begin{tabular}{c c c}
\hline
bytes & throttle\_usecs &  result\\
\hline
200k & 0 & BAD OMEN \& back pressure \\
200k & 10 (100kHz) & BAD OMEN \& back pressure \\
200k & 100 (10kHz) & BAD OMEN \& back pressure\\
200k & 1000 (1kHz) & BAD OMEN \& back pressure \\
200k & 10000 (100Hz) & BAD OMEN \& back pressure \\
200k & 100000 (10Hz) & BAD OMEN \& back pressure\\
\end{tabular}
\caption{We tried to change rates and to use a fixed number of bytes (200kB). Results: we cannot operate with this event size.}
\end{table}\label{tab:upperlimits}
\end{center}
\begin{center}  
\begin{table}[!h]
\centering
\begin{tabular}{c c c}
\hline
bytes & throttle\_usecs &  result\\
\hline
100k & 1000 (1kHz) & BAD OMEN \\
100k & 2000 (500Hz) & BAD OMEN \\
100k & 4000 (250Hz) & BAD OMEN \\
100k & 5000 (200Hz) & BAD OMEN \\
100k & 6000 (166Hz) & OKAY: rate 163Hz \\
100k & 10000 (100Hz) & OKAY: rate 98.5Hz \\
\end{tabular}
\caption{We tried to change rates and to use a fixed number of bytes (100kB). Results: at 20MB/s more or less it gets errors.}
\end{table}\label{tab:upperlimits}
\end{center}
\begin{center}  
\begin{table}[!h]
\centering
\begin{tabular}{c c c}
\hline
bytes & throttle\_usecs &  result\\
\hline
40k & 2000 (500Hz) & BAD OMEN  \\
40k & 3000 (333Hz) & OKAY: rate 320Hz  \\
40k & 5000 (200Hz) & OKAY: rate 195Hz  \\
40k & 10000 (100Hz) & OKAY: rate 98.7Hz \\
\end{tabular}
\caption{We tried to change rates and to use a fixed number of bytes (40k). Results: at 20MB/s more or less it gets errors.}
\end{table}\label{tab:upperlimits}
\end{center}

\begin{center}  
\begin{table}[!h]
\centering
\begin{tabular}{c c c}
\hline
bytes & throttle\_usecs &  result\\
\hline
70k & 5000 (200Hz) & OKAY: rate 195Hz  \\
70k & 4000 (250Hz) & OKAY: rate 242Hz and BAD OMEN after 1 m  \\
70k & 3000 (333Hz) & BAD OMEN  \\

\end{tabular}
\caption{We tried to change rates and to use a fixed number of bytes (70k). Results: at 20MB/s more or less it gets errors.}
\end{table}\label{tab:upperlimits}
\end{center}

We have tried to change also the other variables listed before, but nothing else changed. As we can see it is not possible to run artdaq at rates higher than 20MB/s and this needs to be fixed. 

\end{document}

