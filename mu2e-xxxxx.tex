\documentclass{article}
\usepackage{graphicx,geometry} % Required for inserting images
\geometry{a4paper, top=1.5cm, bottom=1.5cm, left=2cm, right=2cm}

\title{Artdaq DEMO simulation}
\author{Sara Gamba}
\date{\today}

\begin{document}

\maketitle

\section{08/02/24}
We are running an artdaq DEMO simulation: we have 2 board-readers that fill 2 event-builders that have in common one data-logger. There is also a dispatcher which aggregates DQM metrics and presents them to a visualizer application.
In this simulation we are trying to fill the buffer of boardreaders with some numbers, changing frequency between events and then readout.
(XXXX).Eighteen cables (108 links) are used for the Tracker, with four Tracker Readout Controllers sharing each pair of redundant links. The data rate will average 160 MBytes/s per link with two Tracker or three Calorimeter Readout Controllers. We have 40 DAQ nodes, each one with 2 DTCs. Each DTC will have one board-reader, so we expect the minimum input rate in each boardreader is 2kHz more or less. If boardreaders are not able to read this, there could be a bottleneck and the experiment can't do what is expected.
\subsection{Variables/Functions needed for the analysis and changes we have done}
\begin{itemize}
    \item  $max\_fragment\_size\_bytes$: it is the maximal dimension per fragment in the shared memory that is allocated (important that the event size is less than this value);
    \item $AllocateReadoutBuffer(char** buffer)$: it is a function that allocates memory for the buffer (each channel of the tracker will have this buffer). Its value is $uint8\_t(sizeof(demo::ToyFragment::Header) + maxADCchannels\_ \times sizeof(data\_t)))$, where $maxADCchannels\_$ is set at 50M and the $size(data\_t)$ is 2Bytes. Approximately is 100MBytes allocated;
    \item $nADCchannels\_$: it is used where the read bytes are defined, so $bytes\_read = sizeof(demo::ToyFragment::Header) + nADCchannels\_after\_N\_seconds\_ \times sizeof(data\_t)$, so the dimension will be this number $\times$ 2Bytes;
    \item In the function $FillBuffer(buffer,bytes\_read)$ we start filling the first channel after the header in the buffer, with some numbers;
    \item This function is called each $throttle\_ usecs \ \mu$s. Outputs $New FragmentPtrs$ will be added to this container and the function will return True if data-taking should continue.
\end{itemize}
If the artdaq framework works we should see a $GetNext$ Frequency equal to the inverse of the $throttle\_ usecs \ \mu$s value.

Next events are with 2 BR.
\\
BAD OMEN EVERYWHERE.
\section{12/02/24}
\begin{center}  
\begin{table}[!h]
\centering
\begin{tabular}{c c c}
\hline
bytes & throttle &  result\\
\hline
200k & 0 & BAD OMEN \& back pressure \\
200k & 10 (100kHz) & BAD OMEN \& back pressure \\
200k & 100 (10kHz) & BAD OMEN \& back pressure\\
200k & 1000 (1kHz) & BAD OMEN \& back pressure \\
200k & 10000 (100Hz) & BAD OMEN \& back pressure \\
200k & 100000 (10Hz) & BAD OMEN \& back pressure\\
\end{tabular}
\caption{ results.}
\end{table}\label{tab:upperlimits}
\end{center}
\begin{center}  
\begin{table}[!h]
\centering
\begin{tabular}{c c c}
\hline
bytes & throttle &  result\\
\hline
100k & 1000 (1kHz) & BAD OMEN \\
100k & 2000 (500Hz) & BAD OMEN \\
100k & 4000 (250Hz) & BAD OMEN \\
100k & 5000 (200Hz) & BAD OMEN \\
100k & 6000 (166Hz) & OKAY:rate 163Hz \\
100k & 10000 (100Hz) & OKAY:rate 98.5Hz \\
\end{tabular}
\caption{ results. At 20MB/s more or less it gets errors.}
\end{table}\label{tab:upperlimits}
\end{center}
\begin{center}  
\begin{table}[!h]
\centering
\begin{tabular}{c c c}
\hline
bytes & throttle &  result\\
\hline
40k & 2000 (500Hz) & BAD OMEN  \\
40k & 3000 (333Hz) & OKAY:rate 320Hz  \\
40k & 5000 (200Hz) & OKAY:rate 195Hz  \\
40k & 10000 (100Hz) & OKAY:rate 98.7Hz \\
\end{tabular}
\caption{ results. At 20MB/s more or less it gets errors.}
\end{table}\label{tab:upperlimits}
\end{center}

\begin{center}  
\begin{table}[!h]
\centering
\begin{tabular}{c c c}
\hline
bytes & throttle &  result\\
\hline
70k & 5000 (200Hz) & OKAY:rate 195Hz  \\
70k & 4000 (250Hz) & OKAY:rate 242Hz and BAD OMEN after one min  \\
70k & 3000 (333Hz) & BAD OMEN  \\

\end{tabular}
\caption{ results. At 20MB/s more or less it gets errors.}
\end{table}\label{tab:upperlimits}
\end{center}


Error message: 
\begin{itemize}
    \item Bad Omen: Data Buffer has exceeded its size limits.
    \\
    ($seq\_id$=125, $frag\_id$=0, frags=1001/1000, szB=200248048/1048576000), timestamps=124-1124
    \\
    with this type of error the rate is not stable
    \item Back-pressure condition: All Shared Memory buffers have been full for 12.025 s!
\end{itemize}

The same with Shmem instead of TCPSocket.
\end{document}

